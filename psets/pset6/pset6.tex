%%%%%%%%%%%%%%%%%%%%%%%%%%%%%% Preamble
\documentclass{article}
\usepackage{amsmath,amssymb,amsthm,fullpage}
\usepackage[a4paper,bindingoffset=0in,left=1in,right=1in,top=1in,
bottom=1in,footskip=0in]{geometry}
\newtheorem*{prop}{Proposition}
%\newcounter{Examplecount}
%\setcounter{Examplecount}{0}
\newenvironment{discussion}{\noindent Discussion.}{}
\setlength{\headheight}{12pt}
\setlength{\headsep}{10pt}
\usepackage{fancyhdr}
\pagestyle{fancy}
\fancyhf{}
\lhead{CS121 Pset 6}
\rhead{Matt Lim}
\pagenumbering{gobble}
\pagenumbering{gobble}
\begin{document}

%%%%%%%%%%%%%%%%%%%%%%%%%%%%%% Problem 1
\section*{Problem 1}
\begin{description}
    \item[(a)] There are no non-trivial functional dependencies.
    \item[(a)] $F = \{B \rightarrow A\}$
    \item[(a)] $F = \{A \rightarrow B\}$
    \item[(a)] $F = \{A \rightarrow B, B \rightarrow A\}$.
\end{description}

%%%%%%%%%%%%%%%%%%%%%%%%%%%%%% Problem 2
\section*{Problem 2}
\begin{description}
    \item[(a)] Here we will prove that the Union rule is sound. So,
        we want to prove that
        \[ A \rightarrow B \text{ and } A \rightarrow C \implies
        A \rightarrow BC \]
        So, we have that $A \rightarrow B$ and $A \rightarrow C$. Then we have
        that by augmentation on the first depedency, $AC \rightarrow BC$ and
        that by augmentation on the second depedency,
        $AA \rightarrow AC \implies A \rightarrow AC$. Then, using transitivity
        with these two augmentations, we have that $A \rightarrow BC$, as desired.
    \item[(b)] Here we will prove that the Decomposition rule is sound. So,
        we want to prove that
        \[ A \rightarrow BC \text{ and } A \rightarrow B \implies
        A \rightarrow C \]
        So, we have that $A \rightarrow BC$ and $A \rightarrow B$. Then, using
        the reflexivity rule, we have that $BC \rightarrow B$ and
        $BC \rightarrow C$. Using transitivity with these two dependencies and
        the original two dependencies, we
        get that $A \rightarrow B$ and $A \rightarrow C$, as desired.
    \item[(c)] Here we will prove that the Pseudotransitivity rule is sound.
        So, we want to prove that
        \[ A \rightarrow B \text{ and } CB \rightarrow D \implies
        AC \rightarrow D \]
        So, we have that $A \rightarrow B$ and $CB \rightarrow D$. Then, using
        augmentation on the first dependency, we have that $AC \rightarrow BC$.
        Then, using transitivity with $CB \rightarrow D$, we have
        that $AC \rightarrow D$, as desired.
\end{description}

%%%%%%%%%%%%%%%%%%%%%%%%%%%%%% Problem 3
\section*{Problem 3}
\begin{description}
    \item[(a)] We have that:
        \[ A^+ = ABCDE \]
        \[ B^+ = BD \]
        \[ C^+ = C \]
        \[ D^+ = D \]
        \[ E^+ = EABCD \]
        Thus, we have that our candidate keys are $A, E, CD, BC$. $A$ and $E$
        are candidate keys for obvious reasons. $CD$ is a candidate key
        because $CD \rightarrow E$ and $E$ is a superkey and neither $C$
        or $D$ is a superkey. $BC$ is a candidate key because with $B$
        and $C$ you can get $E$ (because $B \rightarrow D$) and $E$ is a superkey
        and neither $B$ or $D$ is a superkey.

        To properly show each candidate key is a superkey for $R$, we can compute
        each keys attribute-set closure.
        So:
        \[ A^+ = ABCDE \]
        We can see this because $A \rightarrow BC$ causes $A^+ = ABC$, then
        $B \rightarrow D$ causes $A^+ = ABCD$, then $CD \rightarrow E$ causes
        $A^+ = ABCDE$.
        \[ E^+ = ABCDE \]
        We can see this because $E \rightarrow A$ and $A$ is a superkey.
        \[ (CD)^+ = ABCDE \]
        We can see this because $CD \rightarrow E$ and $E$ is a superkey.
        \[ (BC)^+ = ABCDE \]
        We can see this because $B \rightarrow D$ causes $(BC)^+ = BCD$, then
        $CD \rightarrow E$ causes $(BC)^+ = ABCDE$ because $E$ is a
        superkey.
    \item[(b)]
        We want to describe all functional dependencies that will appear in the
        closure $F^+$ of $F$. So, we can summarize this as the following.
        First, we have all dependencies
        $\alpha \rightarrow \beta$
        where $\alpha$ is an element from the set of all superkeys
        and $\beta$ is $ABCDE$, and all dependencies generated from this by applying
        the Decomposition Rule. Then we also have all the dependencies in $F$ and
        all trivial dependencies $\alpha \rightarrow \beta$ where $\alpha
        \subseteq \{ABCDE\}$ and $\beta \subseteq A$.
\end{description}

%%%%%%%%%%%%%%%%%%%%%%%%%%%%%% Problem 4
\section*{Problem 4}
Here is a counterexample. Here is a table where
$A \twoheadrightarrow BC$ holds.

\begin{tabular}{ | l | l | c | c | c | }
  \hline
    & $A$ & $B$ & $C$ & $R - (A \cup BC)$ \\ \hline
  $t_1$ & 1 & 2 & 7 & 9 \\
  $t_2$ & 1 & 7 & 5 & 2 \\
  $t_3$ & 1 & 2 & 7 & 2 \\
  $t_4$ & 1 & 7 & 5 & 9 \\
  \hline
\end{tabular}

Now assume to the contrary that $A \twoheadrightarrow B$. Then we have that
$t_1[R - B] = t_4[R - B]$. But we can see here that $t_1[R - B] = 179$ and
that $t_4[R - B] = 159$. So we get a contradiction. So clearly
the original statement is false.

%%%%%%%%%%%%%%%%%%%%%%%%%%%%%% Problem 5
\section*{Problem 5}
\begin{description}
    \item[(a)] Here we wish to compute a canonical cover of
        \[ F = \{ A \rightarrow E, BC \rightarrow D, C \rightarrow A,
        AB \rightarrow D, D \rightarrow G, BC \rightarrow E, D \rightarrow E,
        BC \rightarrow A \]
        So first, we can get rid of $BC \rightarrow E$. This dependency is
        extraneous because $BC \rightarrow A$ and $A \rightarrow E$. So clearly
        these two dependencies and transitivity logically imply the extraneous
        one. Now we have
        \[ F_c = \{ A \rightarrow E, BC \rightarrow D, C \rightarrow A,
        AB \rightarrow D, D \rightarrow G, D \rightarrow E, BC \rightarrow A\} \]
        Secondly, we can get rid of $BC \rightarrow A$. This is extraneous
        because $BC \rightarrow C$ and $C \rightarrow A$. Clearly, these
        two dependencies with transitivity imply the extraneous one.
        Now we have
        \[ F_c = \{ A \rightarrow E, BC \rightarrow D, C \rightarrow A,
        AB \rightarrow D, D \rightarrow G, D \rightarrow E\} \]
        Thirdly, we can add $D \rightarrow GE$ in replacement of
        $D \rightarrow G$ and $D \rightarrow E$. This is because we have
        $D \rightarrow E$ and $D \rightarrow G$, so by the Union rule, we can
        add $D \rightarrow GE$ (and by the Decomposition rule, putting this in
        is logically equivalent to having both $D \rightarrow G$ and
        $D \rightarrow E$. Now we have
        \[ F_c = \{ A \rightarrow E, BC \rightarrow D, C \rightarrow A,
        AB \rightarrow D, D \rightarrow GE \} \]
        Finally, we can get rid of $BC \rightarrow D$. This is because we
        have that $C \rightarrow A$. Applying augmentation to this gives us
        $CB \rightarrow AB$. Then, by transitivity, (and using the dependency
        $AB \rightarrow D$) we have that $CB \rightarrow D$. So that dependency
        is extraneous because other dependencies in $F_c$ logically imply it.
        Thus, finally, we are left with
        \[ F_c = \{ A \rightarrow E, C \rightarrow A, AB \rightarrow D,
        D \rightarrow GE \} \]
        and by looking at it, we can see that it is a valid canonical cover.
    \item[(b)] We have that a candidate key for $R$ is $BC$. Now we will
        compute its attribute-set closure, $(BC)^+$. We can see that
        $C \rightarrow A$ causes $(BC)^+ = BCA$, then $AB \rightarrow D$ causes
        $(BC)^+ = BCAD$, then $D \rightarrow GE$ causes $(BC)^+ = ABCDEG$.
        So we have computed the attribute-set closure for our candidate key.
        Now we must demonstrate that it is a candidate key by computing the
        attribute-set closures of its proper subsets. So, let us begin.
        First we will compute $B^+$. We have that $B^+ = B$; this is clear
        given $F$. Next, we will compute $C^+$. We have that $C \rightarrow A$
        causes $C^+ = CA$, then $A \rightarrow E$ causes $C^+ = CAE$.
        Neither $B^+$ nor $C^+$ is a superkey, so we have that $BC$ is a valid
        candidate key for $R$.
    \item[(c)] We want to decompose $R$ into a BCNF schema. We will do this
        in the following way. We have that $A \rightarrow E$ is not a trivial
        dependency, and that $A$ isn't a superkey for $R$. So we can decompose
        $R$ into $R_1 = (\underline{A}, E)$, $R_2 = (\underline{B},
        \underline{C}, A, D, G)$. Then we have that $D \rightarrow GE$ is not
        a trivial dependency, and that $D$ isn't a superkey for $R_2$. So
        we can decompose further into $R_1 = (\underline{A}, E)$,
        $R_2 = (\underline{D}, G, E)$, $R_3 = (\underline{B}, \underline{C},
        A, D)$. Then we have that $AB \rightarrow D$ isn't a trival dependency,
        and that $AB$ isn't a superkey for $R_3$. So we can decompose further
        into $R_1 = (\underline{A}, E)$, $R_2 = (\underline{D}, G, E)$,
        $R_3 = (\underline{A}, \underline{B}, D)$, $R_4 = \{\underline{B},
        \underline{C}, A)$. Then we have that $C \rightarrow A$ isn't a trivial
        dependency, and that $C$ isn't a superkey for $R_4$. So finally, we
        end up with $R_1 = (\underline{A}, E)$, $R_2 = (\underline{D}, G, E)$,
        $R_3 = (\underline{A}, \underline{B}, D)$, $R_4 = (\underline{C}, A)$,
        $R_5 = (\underline{B}, \underline{C})$. We can now see that there are
        no dependencies that break BCNF. So each of our final relation-schemas
        is in BCNF. More specifically, for $R_1$, the only non-trivial
        dependency $\alpha \rightarrow \beta$ that holds has $\alpha$ as
        the superkey of $R_1$ ($A \rightarrow E$).
        For $R_2$, the only non-trivial
        dependencies $\alpha \rightarrow \beta$ that holds has $\alpha$ as
        the superkey of $R_2$ ($D \rightarrow E$, $D \rightarrow G$).
        For $R_3$, the only non-trivial
        dependency $\alpha \rightarrow \beta$ that holds has $\alpha$ as
        the superkey of $R_3$ ($AB \rightarrow D$).
        For $R_4$, the only non-trivial
        dependency $\alpha \rightarrow \beta$ that holds has $\alpha$ as
        the superkey of $R_4$ ($C \rightarrow A$). And for $R_5$, we
        clearly don't have any dependencies that break BCNF.

        As far as which dependencies in $F_c$ are not preserved by this
        decomposition, we have that it preserves all of them (because each
        dependency appears in one of our decomposed schemas). So basically,
        if $F' = F_1 \cup F_2 \cup \cdots \cup F_5$, we have that
        $F'^+ = F^+$.
    \item[(d)] We want to decompose $R$ into a BCNF schema. We will do this
        in the following way. We have that $A \rightarrow E$ is not a trivial
        dependency, and that $A$ isn't a superkey for $R$. So we can decompose
        $R$ into $R_1 = (\underline{A}, E)$, $R_2 = (\underline{B},
        \underline{C}, A, D, G)$. Then we have that $C \rightarrow A$ is
        not a trivial dependency, and that $C$ is not a superkey for $R_2$.
        So we can decompose further into $R_1 = (\underline{A}, E)$,
        $R_2 = (\underline{C}, A)$, $R_3 = (\underline{B}, \underline{C},
        D, G)$. Then we have that $D \rightarrow G$ is not a trivial
        dependency, and that $D$ is not a superkey for $R_3$. So we can
        decompose further into $R_1 = (\underline{A}, E)$,
        $R_2 = (\underline{C}, A)$, $R_3 = (\underline{D}, G)$,
        $R_4 = (\underline{B}, \underline{C}, D)$. We can now see that this
        is our final answer. This is because we can now see that there are
        no dependencies that break BCNF. So each of our final relation-schemas
        is in BCNF. More specifically, for $R_1$, the only non-trivial
        dependency $\alpha \rightarrow \beta$ that holds has $\alpha$ as
        the superkey of $R_1$ ($A \rightarrow E$).
        For $R_2$, the only non-trivial
        dependencies $\alpha \rightarrow \beta$ that holds has $\alpha$ as
        the superkey of $R_2$ ($C \rightarrow A$).
        For $R_3$, the only non-trivial
        dependency $\alpha \rightarrow \beta$ that holds has $\alpha$ as
        the superkey of $R_3$ ($D \rightarrow G$).
        For $R_4$, the only non-trivial
        dependency $\alpha \rightarrow \beta$ that holds has $\alpha$ as
        the superkey of $R_4$ ($BC \rightarrow D$).

        Now we will specify which dependencies in $F_c$ are not preserved by
        this decomposition. We have that this decomposition does not preserve
        $AB \rightarrow D$ and $D \rightarrow GE$ because these dependencies
        are not included $F'^+$, where $F' = F_1 \cup \cdots \cup F_4$.
    \item[(e)]
        We have that our $F_c = \{A \rightarrow E, C \rightarrow A,
        AB \rightarrow D, D \rightarrow GE\}$. Now, using the 3NF Synthesis
        Algorithm, we can create a 3NF Schema for $R$. Following slide 19
        in lecture $20$, we get
        $R_1 = (\underline{A},E)$, $R_2 = (\underline{C}, A)$,
        $R_3 = (\underline{A}, \underline{B}, D)$, $R_4 = (\underline{D}, G, E)$,
        $R_5 = (\underline{B}, \underline{C})$. We get the first four
        relation schemas from $F_c$, because none of the dependencies in $F_c$
        contain each other completely. Then we get the last one by adding
        in the candidate key for $R$ because no schema in $R_1$ through $R_4$
        contains a candidate key for $R$.
\end{description}

%%%%%%%%%%%%%%%%%%%%%%%%%%%%%% Problem 6
\section*{Problem 6}
\begin{description}
    \item[(a)] Our first candidate key is $course\_id, section\_id, term, year$.
        We can show that this is a superkey by finding the attribute-set closure.
        We have that the attribute-set closure is $\{course\_id, section\_id,
        term, \\ year, meet\_time, room, num\_students, instructor\_id\}$ from the
        second dependency. Then we have that the attribute-set closure is
        $\{course\_id, section\_id, term, year, meet\_time, room, num\_students, \\
        instructor\_id, course\_id\}$ from the third dependency. Then we have
        that the attribute-set closure is everything ($\{course\_id, section\_id,
        term, year, meet\_time, room, num\_students, instructor\_id, course\_id, \\
        dept, units, course\_level\}$) from the first dependency.

        Our second candidate key is $room, meet\_time, term, year$. We can show
        that this is a superkey by finding the attribute-set closure. We have
        that the attribute-set closure is $\{room, meet\_time, term, year, \\
        instructor\_id, course\_id, section\_id\}$ from the third dependency.
        Then we have that the attribute-set closure is
        everything ($\{course\_id, section\_id,
        term, year, meet\_time, room, num\_students, instructor\_id, \\ course\_id,
        dept, units, course\_level\}$) from the first and second dependencies.
    \item[(b)] We have that one canonical cover $F_{c1}$ is as follows:
        \[ \{ course\_id \} \rightarrow \{ dept, units, course\_level \} \]
        \[ \{ course\_id, section\_id, term, year \} \rightarrow
        \{ meet\_time, room, num\_students, instructor\_id \} \]
        \[ \{ room, meet\_time, term, year \} \rightarrow \{ course\_id,
        section\_id \} \]
        We have that the $instructor\_id$ in the third dependency is extraneous
        because it is in the attribute-set closure of $A = \{room, meet\_time,
        term, year\}$. This is true because with the third dependency in
        $F_{c1}$ we have that the attribute-set closure of $A$ is
        $\{room, meet\_time, term, year, course\_id, section\_id\}$. Then, with
        the second dependency in $F_{c1}$, we have that the attribute-set
        closure is $\{room, meet\_time, \\ term, year, course\_id, section\_id,
        num\_students, instructor\_id \}$ ($instructor\_id$ is in there!).

        \vspace{3mm}
        We have that one canonical cover $F_{c2}$ is as follows:
        \[ \{ course\_id \} \rightarrow \{ dept, units, course\_level \} \]
        \[ \{ course\_id, section\_id, term, year \} \rightarrow
        \{ meet\_time, room, num\_students \} \]
        \[ \{ room, meet\_time, term, year \} \rightarrow \{ instructor\_id,
        course\_id, section\_id \} \]
        We have that the $instructor\_id$ in the second dependency is extraneous
        because it is in the attribute-set closure of $A = \{course\_id,
        section\_id, term, year \}$. This is true because with the second
        dependency in $F_{c2}$ we have that the attribute-set closure of $A$ is
        $\{course\_id, section\_id, term, year, meet\_time,\\
        room, num\_students\}$.
        Then, with the third dependency in $F_{c2}$, we have that the attribute-set
        closure is
        $\{course\_id, section\_id, term, year, meet\_time, room,
        num\_students, instructor\_id\}$ ($instructor\_id$ is in there!).

        \vspace{3mm}
        We have that $F_{c1}$ is most appropriate because it makes more sense
        to explicitly and visibly associate $instructor\_id$ with $course\_id$ and
        $section\_id$ than with $room$ and $meet\_time$. This is because
        an instructor is more strongly correlated with a class and a section
        than with a room and a time.
    \item[(c)]
        We will use 3NF here. The schema decomposition that results in using
        this is
        \[ R_1 = (\underline{course\_id}, dept, units, course\_level) \]
        \[ R_2 = (\underline{course\_id, section\_id, term, year}, meet\_time,
            room, num\_students) \]
        \[ R_3 = (\underline{room, meet\_time, term, year}, course\_id,
            section\_id) \]
        The primary keys are underlined, as are the candidate keys. We have
        a foreign key $R_2[course\_id]$ references $R_1[course\_id]$.
        We also have a foreign key $R_3[course\_id]$ references $R_1[course\_id]$.
        We have a foreign key $R_3[course\_id, section\_id, term, year]$
        references $R_2[course\_id, section\_id, term, year]$.

        3NF makes the most since here because in this case, we would rather
        preserve the dependencies than get rid of redudancies (like BCNF does).
        3NF also makes checking constraints faster because it eliminates the
        needs for joins. In this case, since the scale isn't dramatically large
        (it is just a database for courses after all) we are not too hampered
        by redundancies.
\end{description}

%%%%%%%%%%%%%%%%%%%%%%%%%%%%%% Problem 7
\section*{Problem 7}
We have that $email\_id \twoheadrightarrow to\_addr$ is not a trivial
multivalued dependency and $email\_id$ is not a superkey for $R$.
So, we can first decompose this as follows, using the second given dependency:
\[ email\_recipients = (\underline{email\_id, to\_addr}) \]
\[ email\_info = (\underline{email\_id, attachment\_name}, send\_date,
    from\_addr, subject, email\_body, attachment\_body) \]
Then we have that $email\_id, attachment\_name \rightarrow attachment\_body$.
And since if $\alpha \rightarrow \beta$ then $\alpha \twoheadrightarrow \beta$,
then we can decompose this as follows, using the third given dependency:
\[ email\_recipients = (\underline{email\_id, to\_addr}) \]
\[ email\_attachments = (\underline{email\_id, attachment\_name},
    attachment\_body) \]
\[ email\_info = (email\_id, send\_date,
    from\_addr, subject, email\_body, attachment\_name) \]
Then we have that $email\_id$ isn't a superkey for $email\_info$ because
$attachment\_name$ is still in there, so we get non-trivial multivalued
dependencies of the form $\alpha \twoheadrightarrow \beta$ where $\alpha$
is not a superkey of $email\_id$. So then we can decompose this as follows,
using the first given dependency:
\[ email\_recipients = (\underline{email\_id, to\_addr}) \]
\[ email\_attachments = (\underline{email\_id, attachment\_name},
    attachment\_body) \]
\[ email\_info = (\underline{email\_id}, send\_date, from\_addr, subject,
    email\_body) \]
\[ email\_attachment\_names = (\underline{email\_id, attachment\_name}) \]

We have that for $email\_recipients$, $email\_id \twoheadrightarrow to\_addr$
is a trivial multivalued dependency because
$email\_id \hspace{1mm} \cup \hspace{1mm} to\_addr = email\_recipients$. And
all other dependencies that apply are trivial .So
that schema is in 4NF. Then, for $email\_attachments$, every multivalued
dependency $\alpha \twoheadrightarrow \beta$ that applies to it has $\alpha$
as a superkey for $email\_attachments$ ($email\_id, attachment\_name \rightarrow
attachment\_body$).Then, for $email\_info$, every multivalued
dependency $\alpha \twoheadrightarrow \beta$ that applies to it has $\alpha$
as a superkey for $email\_info$, since the only dependencies that apply
have $\alpha = email\_id$ and $email\_id$ is a superkey. Then, we have
that for $email\_attachment\_names$, there are only trival dependencies. So
all our schemas are in 4NF. Note that the last table is actually redudant
so we will not consider it in the following section.

The primary keys are underlined. We have a foreign key
$email\_recipients[email\_id]$ references $email\_info[email\_id]$.
We have a foreign key $email\_attachments[email\_id]$ references
$email\_info[email\_id]$.
\end{document}
