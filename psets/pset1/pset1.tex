%%%%%%%%%%%%%%%%%%%%%%%%%%%%%% Preamble
\documentclass{article}
\usepackage{amsmath,amssymb,amsthm,fullpage}
\usepackage[a4paper,bindingoffset=0in,left=1in,right=1in,top=1in,
bottom=1in,footskip=0in]{geometry}
\newtheorem*{prop}{Proposition}
%\newcounter{Examplecount}
%\setcounter{Examplecount}{0}
\newenvironment{discussion}{\noindent Discussion.}{}
\setlength{\headheight}{12pt}
\setlength{\headsep}{10pt}
\usepackage{fancyhdr}
\pagestyle{fancy}
\fancyhf{}
\lhead{CS121 Pset 1}
\rhead{Matt Lim}
\pagenumbering{gobble}

\def\ojoin{\setbox0=\hbox{$\bowtie$}%
    \rule[-.02ex]{.25em}{.4pt}\llap{\rule[\ht0]{.25em}{.4pt}}}
\def\leftouterjoin{\mathbin{\ojoin\mkern-5.8mu\bowtie}}
    \def\rightouterjoin{\mathbin{\bowtie\mkern-5.8mu\ojoin}}
\def\fullouterjoin{\mathbin{\ojoin\mkern-5.8mu\bowtie\mkern-5.8mu\ojoin}}

\begin{document}
%%%%%%%%%%%%%%%%%%%%%%%%%%%%%% Problem 2.5
\section*{2.5}
\begin{description}
    \item[(a)]
        \[ \Pi_{person\_name}(\sigma_{company\_name=\text{"First Bank
                    Corporation"}} (works)) \]
    \item[(b)]
        \[ \Pi_{person\_name, city}(\sigma_{company\_name=\text{"First Bank
                    Corporation"}} (employee \bowtie works)) \]
    \item[(c)]
        \[ \Pi_{person\_name, city, street} (\sigma_{company\_name=
                \text{"First Bank Corporation"} \wedge salary > 10000}
            (employee \bowtie works)) \]
    \item[(d)]
        \[ \Pi_{person\_name}(employee \bowtie works \bowtie company) \]
    \item[(e)]
        \[ company \div \Pi_{city}(\sigma_{company\_name=\text{"Small Bank
                    Corporation"}}(company)) \]
\end{description}
%%%%%%%%%%%%%%%%%%%%%%%%%%%%%% Problem 2.6
\section*{2.6}
The rewritten query would be
\[ \Pi_{customer\_name, customer\_city, loan\_number, amount}(borrower \bowtie
    loan \bowtie customer) \]
\begin{description}
    \item[(a)]
        Because Jackson is not in the customer relation (under $customer\_name$)
    \item[(b)]
        I would either add Jackson to the customer relation or add a
        $customer\_city$ attribute to the borrower scheme.
    \item[(c)]
        \[ \Pi_{customer\_name, customer\_city, loan\_number, amount}(
            (borrower \bowtie loan) \leftouterjoin customer) \]
\end{description}
%%%%%%%%%%%%%%%%%%%%%%%%%%%%%% Problem 2.7
\section*{2.7}
\begin{description}
    \item[(a)]
        \[ works \leftarrow \Pi_{person\_name, company\_name, (salary * 1.10)}
                (\sigma_{ company\_name=\text{"First Bank Corporation"}}(works))
                \hspace{2mm} \cup \]
        \[ \sigma_{company\_name \neq \text{"First Bank Corporation"}}(works) \]
    \item[(b)]
        \[ temp1 \leftarrow \Pi_{person\_name \text{ as } manager\_name,
                    company\_name, salary}(works) \]
        \[ temp2 \leftarrow temp1 \bowtie manages \]
        \[ temp3 \leftarrow \Pi_{manager\_name \text{ as } person\_name,
                    company\_name, salary}(temp2) \]
        \[ works \leftarrow \Pi_{person\_name, company\_name, (salary * 1.10)}(
                \sigma_{(salary * 1.10) \leq 100000}(temp3)) \hspace{2mm} \cup \]
        \[ \Pi_{person\_name, company\_name, (salary * 1.03)}(\sigma_{(salary * 1.10)
                > 100000}(temp3)) \hspace{2mm} \cup \hspace{2mm} (works - temp3) \]
    \item[(c)]
        \[ works \leftarrow \sigma_{company\_name \neq \text{"Small Bank
                        Corporation"}}( works) \]
\end{description}
%%%%%%%%%%%%%%%%%%%%%%%%%%%%%% Problem 2.8
\section*{2.8}
\begin{description}
    \item[(a)]
        \[ \Pi_{account\_number}(\sigma_{customer\_count > 2}(_{account\_number}
            \mathcal{G}_{count(customer\_name) \text{ as } customer\_count}(
            depositor))) \]
    \item[(b)]
        \[ \Pi_{account\_number}(\sigma_{a.account\_number = b.account\_number
                \wedge b.account\_number = c.account\_number} \]
        \[ _{\wedge a.customer\_name \neq b.customer\_name \wedge
                b.customer\_name \neq c.customer\_name} \]
        \[ _{\wedge a.customer\_name \neq c.customer\_name}(\rho_a(depositor)
            \times \rho_b(depositor) \times \rho_c{depositor})) \]

\end{description}
%%%%%%%%%%%%%%%%%%%%%%%%%%%%%% Problem 2.9
\section*{2.9}
\begin{description}
    \item[(a)]
        \[ temp1 \leftarrow _{company\_name} \mathcal{G}_{count(person\_name)
                \text{ as } person\_count}(works) \]
        \[ max = \mathcal{G}_{max (person\_count) \text{ as }
            max\_count}(temp1) \]
        \[ temp2 = temp1 \times max \]
        \[ \Pi_{company\_name}(\sigma_{person\_count = max\_count}(temp2)) \]
    \item[(b)]
        \[ temp1 \leftarrow _{company\_name} \mathcal{G}_{sum(salary)
                    \text{ as } payroll}(works) \]
        \[ min = \mathcal{G}_{min (payroll) \text{ as } min\_payroll}(temp1) \]
        \[ temp2 = temp1 \times min \]
        \[ \Pi_{company\_name}(\sigma_{payroll = min\_payroll}(temp1)) \]
    \item[(c)]
        \[  first\_avg \leftarrow \mathcal{G}_{avg
                    (salary) \text{ as } first\_bank\_avg\_salary}
                (\sigma_{company\_name
                =\text{"First Bank Corporation"}}(works)) \]
        \[ avg\_salaries \leftarrow _{company\_name} \mathcal{G}_{avg
                    (salary) \text { as } avg\_salary}(works) \]
        %\[ first\_avg \leftarrow \Pi_{avg\_salary \text{ as }
                %first\_bank\_avg\_salary}(\sigma_{company\_name=\text{
                        %"First Bank Corporation"}}(avg\_salaries)) \]
        \[ temp1 \leftarrow avg\_salaries \times first\_avg \]
        \[ \Pi_{company\_name}(\sigma_{avg\_salary > first\_bank\_avg\_salary}(
            avg\_salaries)) \]
\end{description}
%%%%%%%%%%%%%%%%%%%%%%%%%%%%%% Problem 5
\section*{Relational Division Operations}
\begin{description}
    \item[(a)]
        Let us label the following relations in the definition
        of relational division as follows
        \begin{itemize}
            \item $a = \Pi_{R-S}(r) \times s$
            \item $b = \Pi_{R-S, S}(r)$
            \item $c = \Pi_{R-S}(a - b)$
            \item $d = \Pi_{R-S}(r)$
        \end{itemize}

        Now for the explanation. $a$ is all possible combinations of names and
        foods, and $b$ is just $monkey\_likes$, but with the attributes ordered
        to be identical
        to that produced by the expression ($\Pi_{R-S}(r) \times s$). Given this,
        we have that $c$ contains all the names that do not have an entry for
        every food in $monkey\_foods$. This is because if a name does have an
        entry for every food in $monkey\_foods$,
        then all those tuples will cancel with the tuples from $a$
        (which contains all the combinations of names and foods). Then we just
        subtract these names from $d$, which just projects the names of the
        monkeys from $r$. So clearly, we are left only with the names of the
        monkeys that \textit{do} have an entry for every food in $s$.

        Now to explain more specifically why Guenter would appear in the result.
        Guenter has an entry for every food in $monkey\_foods$. In other words,
        all possible combinations of Guenter and food are present in
        $monkey\_likes$. So when we do $(a-b)$, no
        tuples will have the name Guenter, because all the rows with his name
        will subtract out. Note that the tuple (Guenter, tofu) doesn't matter
        here, because tofu does not appear in the $food$ column of $monkey\_foods$.
        All that does matter is that Guenter has an entry for every food in
        $monkey\_foods$, so that the subtraction $a-b$ kills his name.
        So then, when we project the names in $c$, then subtract
        $c$ from $d$, we won't subtract the name Guenter, and Guenter will appear
        in the final result.
    \item[(b)]
        \[ r \div _{E} s = r \div s - \Pi_{R-S}(\Pi_{R-S,S}(r) -
            (\Pi_{R-S}(r) \times s)) \]
        We can see that this works for the following reason.
        The second subtraction subtracts the Cartesian product from the reordered
        first relation $r$. This leaves only the tuples from $r$ that
        do not appear in the Cartesian product of $\Pi_{R-S}(r)$ and $s$.
        In other words, this leaves the tuples that do not match the contents
        of $s$. So then we can just subtract them from the regular division and
        get our exact division.

        In terms of the example, the right side of the second subtraction gives
        all combinations of names and foods. So when we subtract this from the left
        side of the second subtraction, we are left only with the names and
        foods that are not valid, i.e. those names that are paired with
        foods not in $monkey\_foods$.
    \item[(c)]
        \[ \Pi_{R-S}(_{R-S} \mathcal{G}_{count(S) \text{ as } num}(r \bowtie s)
            \bowtie \mathcal{G}_{count(S) \text{ as } num}(s)) \]
\end{description}
%%%%%%%%%%%%%%%%%%%%%%%%%%%%%% Problem 5
\section*{Query Optimization and Equivalence Rules}
\begin{description}
    \item[(a)]
        They are equivalent.

        This is because when we select before we group, we
        select a set of tuples based on a predicate $\theta$ using only attributes
        from $A$. So when we select using $\theta$, we select sets of tuples
        that would be groups. That is, if we select a tuple $t$ that would be in
        a group $g$, we also select all other tuples in $g$. This is because
        all tuples in $g$ have the same grouping attributes, and our predicate
        $\theta$ only uses attributes from $A$, the grouping attributes.

        The other way around, we group, then select. So this way,
        we select sets of tuples that are groups.

        So overall, we select the same rows either way, since in both ways
        we select the same groups of tuples.
    \item[(b)]
        They are not equivalent. Here is a counterexample.

        \begin{center}
            Relation $r$
            \begin{tabular}{ | l | l | p{5cm} |}
                \hline
                \textbf{Person} & \textbf{Rating} \\ \hline
                Sid & 0 \\ \hline
                Sid & 1 \\ \hline
                Connor & 2 \\ \hline
            \end{tabular}
        \end{center}

        \begin{center}
            Relation $s$
            \begin{tabular}{ | l | l | p{5cm} |}
                \hline
                \textbf{Person} & \textbf{Rating} \\ \hline
                Sid & 0 \\ \hline
                Matt & 3 \\ \hline
            \end{tabular}
        \end{center}

        We have that $\Pi_{A}(r-s)$, if $A$ projects on \textbf{Person}, would
        project Sid, Connor, and Matt, while $\Pi_{A}(r) - \Pi_{A}(s)$ would project
        only Connor and Matt.
    \item[(c)]
        They are not equivalent. Here is a counterexample.

        \begin{center}
            Relation $r$
            \begin{tabular}{ | l | l | p{5cm} |}
                \hline
                \textbf{a} & \textbf{b1} \\ \hline
                1 & 2 \\ \hline
            \end{tabular}
        \end{center}

        \begin{center}
            Relation $s$
            \begin{tabular}{ | l | l | p{5cm} |}
                \hline
                \textbf{a} & \textbf{b2} \\ \hline
                0 & 3 \\ \hline
            \end{tabular}
        \end{center}

        \begin{center}
            Relation $t$
            \begin{tabular}{ | l | l | p{5cm} |}
                \hline
                \textbf{a} & \textbf{b3} \\ \hline
                1 & 4 \\ \hline
            \end{tabular}
        \end{center}

        We have that $(r \leftouterjoin s) \leftouterjoin t$ gives us a
        relation with a tuple $(1,2,null,4)$, while $r \leftouterjoin (
            s \leftouterjoin t)$ gives us a relation with a tuple
                $1,2,null,null$.
    \item[(d)]
        They are equivalent. This is because $\theta$ is a predicate that uses
        attributes only from $r$, and the left outer join keeps all the tuples
        from $r$, which means that we will select the same tuples
        if we select from $r$ and then left outer join with $s$ as if we
        left outer join $r$ with $s$ and then select tuples. In other words,
        $r \leftouterjoin s$ includes every tuple from $r$ in one of its tuples.
        So selecting on $\theta$, which only uses attributes from $r$,
        selects the same tuples whether it is applied before or after the join.
    \item[(e)]
        They are not equivalent. Here is a counterexample.

        \begin{center}
            Relation $r$
            \begin{tabular}{ | l | p{5cm} |}
                \hline
                \textbf{Store} \\ \hline
                Walmart \\ \hline
                Costco \\ \hline
            \end{tabular}
        \end{center}

        \begin{center}
            Relation $s$
            \begin{tabular}{ | l | l | p{5cm} |}
                \hline
                \textbf{Store} & \textbf{Product} \\ \hline
                Walmart & Boxes \\ \hline
                Costco & Bags \\ \hline
            \end{tabular}
        \end{center}

        We have that $\sigma_{\theta}(r \leftouterjoin s)$, if $\theta$ is the
        predicate $Product = \text{"Boxes"}$, is just the tuple Walmart,
        Boxes. On the other hand, we have that $r \leftouterjoin
        \sigma_{\theta}(s)$ has the tuple Walmart, Boxes and the tuple
        Costco, null.
\end{description}
\end{document}
